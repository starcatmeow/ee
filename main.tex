\documentclass{article}

\title{Evaluating the Relevancy of JFIF Image Compression on the Internet}
\author{Uzen}
\date{Jul 22}

\usepackage{indentfirst}

\parskip=1em

\begin{document}

\maketitle

{\LARGE \tt this is a DRAFT!!!!!!}

\tableofcontents
\newpage

\section{Introduction}

Compression is the cornerstone of the internet.
Behind the millions of bytes flowing between computers everyday around the world, compression algorithms are behind every single data packet.
Compression algorithms are tools that can seemingly magically shrink a digital document, sometimes until they are a tenth of their size.
Due the many uses of compression algorithms, they have been used in many different fields.
From shrinking large backups for storage to speeding up the transfer of data across the web, compression algorithms are crucial to modern computing.

One specific field that I was interested in was website hosting.
A personal website can be a good representation of themselves, similar to a more polished social media page.
For a period of time, I was quite interested in starting my own website, renting virtual servers from Alison to start my project.
However, I quickly ran into an issue.
In the modern times, technology has progressed at an exponential rate, storage and bandwidth speeds are increasing at an extraordinary rate.
It is common to see computers with hundreds of gigabytes of storage and gigabit connection to the internet.
So it was quite a surprise for me when I found out that adding bandwidth and storage to my server costed money.
A small, affordable server would only come with 10 to 50 gigabytes of storage a measly 1 megabit bandwidth connection (or purchase bandwidth by the gigabit used).

Due to this limitation, the website I made was slow.
I couldn't upload many images or videos because the server would run out of storage, or make the website load at a snail's pace.
This made me very careful about the files I upload, especially the sizes of image and videos.
My curiosity eventually lead me to image compression.
When an image is stored, there's different file types:\texttt{~.png,~.jpg,~.gif}.
These are actually different methods of storing images, using different methods image compression techniques.
When creating my website, I had to choose what method of image compression to use.
Each different method has its advantages and disadvantages.

One format of compression I am very interested in was JFIF compression (commonly found in JPEG files).
JFIF uses a mixture of clever math and the characteristics of the human eye to shrink photos.
It is commonly used in digital cameras.
However, JFIF was originally invented over 30 years ago (August 1990\footnote{Add citation here}), and have not account for the many modern technological advances.
For example, JFIF is bad at dealing with digital art and digital text due to limitation of the compression techniques.
In the recent years, newer and more efficient image compression algorithms have been created, such as Webp. 
In fact, JPEG has released a newer version of JFIF, called JPEG-2000.
Some has even used Artificial Intelligence to enhance compression rates\footnote{https://ai.googleblog.com/2016/09/image-compression-with-neural-networks.html \textbf{setup biblatex :/}}.

Nevertheless, JFIF compression still remains as a dominant image compression technique.
This paper aims to compare JFIF with modern image compression formats to evaluate whether it is still a competitive algorithm.
The results of the paper would help website creators decide the best format to store images on their websites.

\section{Background Information}

\subsection{Lossless and Lossy Compression}

\subsection{Commonly Used Types of Compression}

\section{Background Information}

\end{document}
